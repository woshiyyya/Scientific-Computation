\documentclass[12pt]{article}
\usepackage[utf8]{inputenc}
\usepackage{CJKutf8}
\usepackage{amsmath}
\usepackage{setspace}
\usepackage{esint}
\title{Scientific Computation}
\author{Xiao Yunxuan}
\date{November 2017}

\begin{document}
\begin{CJK*}{UTF8}{gbsn}
\begin{spacing}{1.5}
\maketitle
\newpage

\section{正交多项式}
    \subsection{勒让德多项式}
        \subsubsection{形式}
        勒让德多项式是带权$\rho(x) = 1$的正交函数族,          是$\{1,x,x^2,\cdots,x^n,\cdots \}$通过正交化得到的,自然的次数从1~n
        $$
        \left\{
            \begin{aligned}
               &P_0(x) = 1\\
               &P_n(x) = \frac{1}{2^n n!}\frac{d^n}{dx^n}(x^2 - 1)^n
            \end{aligned}
        \right.
        $$
        归一化后形式:\\
        $$\widetilde{P_n}(x) = \frac{n!}{(2n)!}\frac{d^n}{dx^n}(x^2 - 1)^n = \frac{2^n(n!)^2}{(2n)!}P_n(x)$$
        \subsubsection{性质}
        性质1:正交性
        $$
        \int_{-1}^{1} P_n(x)P_m(x)dx =
        \left \{
            \begin{aligned}
                &0,    &m\not=n; \\
                &\frac{2}{2n+1},  &m=n;
            \end{aligned}
        \right.
        $$

\section{数值积分}
    \subsection{梯形算法外推化}
        将区间[a,b]n等分,再将n个区间二等分,得到$\{x_k,x_{k +\frac{1}{2}}\}$ \\
        使用复合梯形公式得到:
        \[
            I = \frac{h}{4}[f(x_k) + f(x_k+\frac{1}{2}]
        \]


\end{spacing}
\end{CJK*}
\end{document}
