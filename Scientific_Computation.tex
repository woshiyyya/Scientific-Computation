\documentclass[12pt]{article}
\usepackage[utf8]{inputenc}
\usepackage{CJKutf8}
\usepackage{amsmath}
\usepackage{setspace}
\usepackage{esint}
\usepackage{geometry}
\title{Scientific Computation}
\author{Xiao Yunxuan}
\date{November 2017}
\geometry{left = 2cm,right = 2cm,bottom = 2.5cm,top = 2.5cm}


\begin{document}
\begin{CJK*}{UTF8}{gbsn}
\begin{spacing}{1.5}
\maketitle
\newpage

\section{正交多项式}
    \subsection{勒让德多项式}
        \subsubsection{形式}
        勒让德多项式是带权$\rho(x) = 1$的正交函数族,          是$\{1,x,x^2,\cdots,x^n,\cdots \}$通过正交化得到的,自然的次数从$1-n$
        $$
        \left\{
            \begin{aligned}
               &P_0(x) = 1\\
               &P_n(x) = \frac{1}{2^n n!}\frac{d^n}{dx^n}(x^2 - 1)^n
            \end{aligned}
        \right.
        $$  
        归一化后形式:\\
        $$\widetilde{P_n}(x) = \frac{n!}{(2n)!}\frac{d^n}{dx^n}(x^2 - 1)^n = \frac{2^n(n!)^2}{(2n)!}P_n(x)$$
        \subsubsection{性质}
        性质1:正交性
        $$
        \int_{-1}^{1} P_n(x)P_m(x)dx =  
        \left \{
            \begin{aligned}
                &0,    &m\not=n; \\
                &\frac{2}{2n+1},  &m=n;
            \end{aligned}
        \right.
        $$
        \subsubsection{应用}
        1.勒让德多项式$P_n$零点作为n-rank Gauss-Legendre 求积公式插值点

\newpage
\section{数值积分}
\subsection{代数精度}
    若某种求积公式对次数$\leq m$的多项式准确成立 \\
    $$\int_a^b f(x)dx = \sum_{k=0}^{n}A_kf(x_k)$$  \\
    则其具有m次代数精度\\
    \textbf{直白表述}:即能找到确定的插分点$\{x_k\}$及其对应的系数$\{A_k\}$,使得其线性组合能精确表示一组基$\{1,x,x^2,\cdots,x^m\}$,则其必对$P_m$精确成立。\\ 
    \\
    化归为解线性方程组:\\
    $$
    \left \{
    \begin{aligned}
        &\sum A_k = b-a\\
        &\sum A_k x_k = \frac{1}{2}(b^2-a^2)\\
        &\qquad    \vdots \\
        &\sum A_k x^m_k = \frac{1}{m+1}(b^{m+1}-a^{m+1})\\
    \end{aligned}
    \right.
    $$
    
    要刻画一个$n$次多项式,至少需要$n+1$个点,加上$n+1$个系数组成的$2n+2$个未知量的方程组\\
    1.一般会告诉插分点,带进去解系数$A_k$\\
    2.继续向下验证是否对$x^{m+1}$也成立,直到推到最大代数精度\\\\
    \textbf{\underline{PS: n阶求积公式——n+1插值点——最少n代数精度——对$P_n$精确成立}}

\newpage
\subsection{插值型求积公式}
    通过插值点$a\leq x_0 < x_1 < \cdots < x_n \leq b$ 得到插值函数$L_n(x)$,以插值函数的积分近似代替I。\\
    (梯形公式与中矩形公式均是特殊的插值求积公式)\\
    $$
    \begin{aligned}
        \int L_n(x)dx &= \int \sum f(x_k)l_k(x)dx  \\
                      &= \sum f(x_k) \int l_k(x)dx \\
                      &= \sum  f(x_k)A_k
    \end{aligned}
    $$
    故系数即插值基函数的积分:
    $$A_k = \int_a^bl_k(x)dx $$
    求积公式的余相:
    $$R[f] = \int[f(x) - L_n(x)] = \int R_n(x)
            = \int \frac{f^{(n+1)}(\xi)}{(n+1)!}\omega_{n+1}(x)
    $$
    1.对于 $\quad \forall f(x) \in P_n(x): \quad  R[f] = 0$精确成立\\
    2.对于 $\quad \forall f(x) \notin P_n(x):\quad$有泛化公式:
    $$
        R[f] = \int_a^b f(x)dx - \sum_{k=0}^n A_k f(x_k) = \boldsymbol{K}f^{(n+1)}(\xi)
    $$
    据此可求出\textbf{$n+1$阶函数求积余项公式} (WARNING:此时$f(x) = x^{n+1}$)\\
    $$    \boldsymbol{K} = \frac{1}{(n+1)!}[\int_a^b x^{n+1}dx - \sum_{k=0}^n A_k x_k^{n+1}]
    $$
\newpage
\subsection{Newton-Cotes Formula}
\textbf{单区间:}等距型插值求积公式
\subsubsection{Trapezoidal Formula}
    $$n = 1,h = b-a$$
    $$I \approx \big(\frac{b-a}{2}\big)[f(x_0)+f(x_1)]$$
    $$R[f] = -\frac{(b-a)^3}{12}f^{(2)}(\xi) \qquad Accuracy = 1$$
\subsubsection{Simpson Formula}
    $$n = 2,h = \frac{b-a}{2}$$
    $$I \approx \frac{(b-a)}{6}[f(x_0) + 4f(x_1)+f(x_2)]$$
    $$R[f] = -\frac{b-a}{180}\big(\frac{b-a}{4}\big)^4f^{(4)}(\xi) \qquad Accuracy = 3$$
\subsubsection{Newton Formula}
    $$n = 3,h = \frac{b-a}{3}$$
    $$I \approx \frac{3h}{8}[f(x_0)+3f(x_1)+3f(x_2)+f(x_3)]$$
    $$R[f] = -\frac{(b-a)}{80}\big(\frac{b-a}{3}\big)^4f^{(4)}(\xi) \qquad Accuracy = 3$$
\subsubsection{Cotes Formula}
    $$n = 4,h = \frac{b-a}{4}$$
    $$I \approx \frac{2h}{45}[7f(x_0)+32f(x_1)+12f(x_2)+32f(x_3) +7f(x_4)]$$
    $$R[f] = -\frac{2(b-a)}{945}\big(\frac{b-a}{4}\big)^6f^{(6)}(\xi) \qquad Accuracy = 5$$
\textbf{规律:偶次阶求积公式具有n+1阶精度。}
\newpage
\textbf{定理}\\
    证明偶次阶求积公式具有n+1阶精度:\\
    我们只需证明n = Even Number时,Newton-Cotes 公式对$f = x^{n+1}$余项为0\\
    引入变换$x = a + th$
    $$
    \begin{aligned}
        R[f] &= \int_a^b\frac{f^{(n+1)}(\xi)}{(n+1)!}\omega_{n+1}(x)\\
             &= \int_a^b \omega_{n+1}(x)\\
             &= h^{n+1}\int_0^n \prod_{j=0}^n(t-j)dht\quad(Alter: t=u+\frac{n}{2})\\  
             &= h^{n+2}\int_{-\frac{n}{2}}^{\frac{n}{2}}\prod_{j=0}^n(u+\frac{n}{2}-j)du \quad (Moving\_towards\_left)\\
             &= 0 \quad (Odd\_function)
    \end{aligned}
    $$
    
\newpage
\subsection{复合求积公式}
\textbf{多区间}:小区间求和插值公式\\
注意:此处n(小区间等分个数)非彼处n(单区间等分个数)\\
      此处h(小区间步长)非彼处h(单区间步长)\\
\subsubsection{复合Trapezoidal公式}
    $$
    \begin{aligned}
        I &= T_n + R_n = \frac{h}{2}\sum_{k=0}^{n-1}[f(x_k) + f(x_{k+1})] + R_n \\
          &= \frac{h}{2}[f(a) + 2\sum_{k=1}^{n-1}f(x_k) + f(b)] + R_n\\
    \end{aligned}
    $$
    $$
    \begin{aligned}
        Where \quad R_n &= \sum_{k=0}^{n-1}\big[-\frac{h^3}{12}f^{(2)}(\eta_k)\big] = -\frac{h^3}{12}n*f^{(2)}(\eta) \\
        &= -\frac{b-a}{12}h^2f^{(2)}(\eta) \qquad \boldsymbol{Error = O(h^2)}
    \end{aligned} 
    $$
    
\subsubsection{复合Simpson公式}
    $$
    \begin{aligned}
        I &= T_n + R_n = \frac{h}{6}\sum_{k=0}^{n-1}[f(x_k)+ 4f(x_{k+\frac{1}{2}}) + f(x_{k+1})] + R_n \\
          &= \frac{h}{2}[f(a) + 4\sum_{k=1}^{n-1}f(x_{k+\frac{1}{2}}) + 2\sum_{k=1}^{n-1}f(x_k) + f(b)] + R_n\\
    \end{aligned}
    $$
    $$
    \begin{aligned}
        Where \quad R_n &= \sum_{k=0}^{n-1}\big[-\frac{h}{180}\big(\frac{h}{2}\big)^4f^{(4)}(\eta_k)\big] = -\frac{h}{180}\big(\frac{h}{2}\big)^4n*f^{(4)}(\eta) \\
        &= -\frac{b-a}{180}\big(\frac{h}{2}\big)^4f^{(4)}(\eta) \qquad \boldsymbol{Error = O(h^4)}
    \end{aligned}
    $$
    
\newpage
\subsection{梯形算法外推化}
        将区间[a,b]n等分,再将n个区间二等分,得到$\{x_k,x_{k +\frac{1}{2}}\}$ \\
        使用复合梯形公式得到:
        \[
            I = \frac{h}{4}[f(x_k) + f(x_k+\frac{1}{2})]
        \]   
\subsection{Richardson 加速}
    梯形公式余项为$O(h^2)$阶,可展开成如下形式
    $$
        T(h) = I + a_1h^2 +a_2h^4+\cdots+ a_kh^{2k} + \cdots
    $$
    令$h <= h/2$:
    $$
        T(\frac{h}{2}) = I + a_1\frac{h^2}{4} +a_2\frac{h^4}{16}+\cdots+ a_k(\frac{h}{2})^{2k} + \cdots
    $$
    $4*(2) - (1)$得:
    $$
        S(h) = \frac{4T(\frac{h}{2})-T(h)}{3} = I + c_1h^4 + c_2h^6 + \cdots
    $$    
    误差降低到$O(h^4)$\\
    由梯形公式$T(h)\&T(h/2)$导出的$S(h)$恰是辛普森公式!!\\
    不断继续下去便形成了Romberg算法。\\
    \textbf{注意:}若使用Simpson公式外推,误差为$O(h^4)$阶,相应导出公式变为:
    $$
        C(h) = \frac{16S(\frac{h}{2})-S(h)}{15} 
    $$
\newpage
\subsection{Romberg Algorithm}
    1.k等分区间梯形公式算$T_0^{(k)}$   \\
    2.依次使用$T_{m-1}^{(k+1)} T_{m-1}^{(k)}$计算一横排$T_m^{(k)}$\\
    3.比较相邻对角元素$T_k^{(k)} T_{k-1}^{(k-1)}$与$\epsilon$\\
    4.停止or继续二分
    \\\\
    \textbf{思考:Richardson加速为什么更优呢?直接用n阶Newton-Cotes公式不行吗?}\\
    由于我们不知道k等分后计算出的结果是否可以达到误差范围,这样会造成我们不断k+1等分,重复计算大量值,且公式繁杂,每个积分公式系数都不同\\
    而使用龙贝格方法优点为:\\
    1.利用了之前运算结果,每次只需算k等分梯形公式值,利用前面结果,通过相同公式反复迭代,方便简洁稳定性好。\\
    2.将先验误差转化为后验误差,根据结果调整计算过程,节约计算量。\\
    3.加速算法,每次步长指数倍缩小,收敛更快。

\newpage
\subsection{Gauss 求积公式}
    我们知道,插值型求积公式最少精度为n,偶阶时精度为n+1,然而适当选取节点可以将精度提高到2n+1,这些点叫Gauss点\\
    \textbf{定义式:}研究带权积分何时精确成立
    $$
        \int_a^bf(x)\rho(x)dx \approx \sum_{k=0}^nA_kf(x_k) 
    $$
    观察余项:
    $$
        R[f] = \frac{1}{(n+1)!}\int_a^bf^{(n+1)}(\xi)\omega_{n+1}(x)\rho(x)dx
    $$
    只需要证明f为2n+1次多项式时,选Gauss点做插值点使得$R[q] = 0$成立(q为n次多项式)即可\\
    \textbf{左推右:}f的n+1阶导为n次,乘上$\omega_{n+1}$为2n+1次,积分为0,OK\\
    \textbf{右推左:}
    $$
        f = p(x)\omega_{n+1}(x) + q(x)
    $$
    因R = 0故
    $$
        \int_a^bf(x)\rho(x)dx = \int_a^bq(x)\rho(x)dx 
    $$
    q(x)为n次多项式,精确成立
    $$
        \int_a^bq(x)\rho(x)dx = \sum_{k=0}^nA_kq(x_k) 
    $$
    取高斯点时,$f(x_k) = q(x_k)$故
    $$
        \int_a^bf(x)\rho(x)dx = \int_a^bq(x)\rho(x)dx = \sum_{k=0}^nA_kf(x_k) 
    $$
    
    \textbf{总结:}所以选择正交多项式的零点做插值点,这样$\omega_{n+1}$就是归一化的正交多项式,就可以对任意n次多项式带权积分为0,保证对2n+1阶的f积分精确成立\\
    其中系数$A_k$与f无关:
    $$
        A_k = \int_a^bl_k(x)\rho(x)dx
    $$
    
    
\end{spacing}        
\end{CJK*}
\end{document}
