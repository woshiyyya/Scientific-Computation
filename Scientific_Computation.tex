\documentclass[12pt]{article}
\usepackage[utf8]{inputenc}
\usepackage{CJKutf8}
\usepackage{amsmath}
\usepackage{setspace}
\usepackage{esint}
\usepackage{geometry}
\title{Scientific Computation}
\author{Xiao Yunxuan}
\date{November 2017}
\geometry{left = 2cm,right = 2cm,bottom = 2.5cm,top = 2.5cm}


\begin{document}
\begin{CJK*}{UTF8}{gbsn}
\begin{spacing}{1.5}
\maketitle
\newpage

\section{正交多项式}
    \subsection{勒让德多项式}
        \subsubsection{形式}
        勒让德多项式是带权$\rho(x) = 1$的正交函数族,          是$\{1,x,x^2,\cdots,x^n,\cdots \}$通过正交化得到的,自然的次数从$1-n$
        $$
        \left\{
            \begin{aligned}
               &P_0(x) = 1\\
               &P_n(x) = \frac{1}{2^n n!}\frac{d^n}{dx^n}(x^2 - 1)^n
            \end{aligned}
        \right.
        $$  
        归一化后形式:\\
        $$\widetilde{P_n}(x) = \frac{n!}{(2n)!}\frac{d^n}{dx^n}(x^2 - 1)^n = \frac{2^n(n!)^2}{(2n)!}P_n(x)$$
        \subsubsection{性质}
        性质1:正交性
        $$
        \int_{-1}^{1} P_n(x)P_m(x)dx =  
        \left \{
            \begin{aligned}
                &0,    &m\not=n; \\
                &\frac{2}{2n+1},  &m=n;
            \end{aligned}
        \right.
        $$
        \subsubsection{应用}
        1.勒让德多项式$P_n$零点作为n-rank Gauss-Legendre 求积公式插值点

\newpage
\section{数值积分}
\subsection{代数精度}
    若某种求积公式对次数$\leq m$的多项式准确成立 \\
    $$\int_a^b f(x)dx = \sum_{k=0}^{n}A_kf(x_k)$$  \\
    则其具有m次代数精度\\
    \textbf{直白表述}:即能找到确定的插分点$\{x_k\}$及其对应的系数$\{A_k\}$,使得其线性组合能精确表示一组基$\{1,x,x^2,\cdots,x^m\}$,则其必对$P_m$精确成立。\\ 
    \\
    化归为解线性方程组:\\
    $$
    \left \{
    \begin{aligned}
        &\sum A_k = b-a\\
        &\sum A_k x_k = \frac{1}{2}(b^2-a^2)\\
        &\qquad    \vdots \\
        &\sum A_k x^m_k = \frac{1}{m+1}(b^{m+1}-a^{m+1})\\
    \end{aligned}
    \right.
    $$
    
    要刻画一个$n$次多项式,至少需要$n+1$个点,加上$n+1$个系数组成的$2n+2$个未知量的方程组\\
    1.一般会告诉插分点,带进去解系数$A_k$\\
    2.继续向下验证是否对$x^{m+1}$也成立,直到推到最大代数精度\\\\
    \textbf{\underline{PS:n阶求积公式——n+1插值点——最少n代数精度——对$P_n$精确成立}}

\newpage
\subsection{插值型求积公式}
    通过插值点$a\leq x_0 < x_1 < \cdots < x_n \leq b$ 得到插值函数$L_n(x)$,以插值函数的积分近似代替I。\\
    (梯形公式与中矩形公式均是特殊的插值求积公式)\\
    $$
    \begin{aligned}
        \int L_n(x)dx &= \int \sum f(x_k)l_k(x)dx  \\
                      &= \sum f(x_k) \int l_k(x)dx \\
                      &= \sum  f(x_k)A_k
    \end{aligned}
    $$
    故系数即插值基函数的积分:
    $$A_k = \int_a^bl_k(x)dx $$
    求积公式的余相:
    $$R[f] = \int[f(x) - L_n(x)] = \int R_n(x)
            = \int \frac{f^{(n+1)}(\xi)}{(n+1)!}\omega_{n+1}(x)
    $$
    1.对于 $\quad \forall f(x) \in P_n(x): \quad  R[f] = 0$精确成立\\
    2.对于 $\quad \forall f(x) \notin P_n(x):\quad$有泛化公式:
    $$
        R[f] = \int_a^b f(x)dx - \sum_{k=0}^n A_k f(x_k) = \boldsymbol{K}f^{(n+1)}(\xi)
    $$
    据此可求出\textbf{$n+1$阶函数求积余项公式}(WARNING:此时$f(x) = x^{n+1}$)\\
    $$    \boldsymbol{K} = \frac{1}{(n+1)!}[\int_a^b x^{n+1}dx - \sum_{k=0}^n A_k x_k^{n+1}]
    $$
\newpage
\subsection{Newton-Cotes Formula}
等距型插值求积公式
\subsubsection{Trapezoidal Formula}
    $$n = 1,h = b-a$$
    $$I \approx \frac{h}{2}[f(x_0)+f(x_1)]$$
    $$R[f] = -\frac{h^3}{12}f^{(2)}(\xi) \qquad Accuracy = 1$$
\subsubsection{Simpson Formula}
    $$n = 2,h = \frac{b-a}{2}$$
    $$I \approx \frac{h}{3}[f(x_0) + 4f(x_1)+f(x_2)]$$
    $$R[f] = -\frac{h^5}{90}f^{(4)}(\xi) \qquad Accuracy = 3$$
\subsubsection{Newton Formula}
    $$n = 3,h = \frac{b-a}{3}$$
    $$I \approx \frac{3h}{8}[f(x_0)+3f(x_1)+3f(x_2)+f(x_3)]$$
    $$R[f] = -\frac{3h^5}{80}f^{(4)}(\xi) \qquad Accuracy = 3$$
\subsubsection{Cotes Formula}
    $$n = 4,h = \frac{b-a}{4}$$
    $$I \approx \frac{2h}{45}[7f(x_0)+32f(x_1)+12f(x_2)+32f(x_3) +7f(x_4)]$$
    $$R[f] = -\frac{8h^7}{945}f^{(6)}(\xi) \qquad Accuracy = 5$$
\textbf{规律:偶次阶求积公式具有n+1阶精度。}
\newpage
\textbf{定理}\\
    证明偶次阶求积公式具有n+1阶精度:\\
    我们只需证明n = Even Number时,Newton-Cotes 公式对$f = x^{n+1}$余项为0\\
    引入变换$x = a + th$
    $$
    \begin{aligned}
        R[f] &= \int_a^b\frac{f^{(n+1)}(\xi)}{(n+1)!}\omega_{n+1}(x)\\
             &= \int_a^b \omega_{n+1}(x)\\
             &= h^{n+1}\int_0^n \prod_{j=0}^n(t-j)dht\quad(Alter: t=u+\frac{n}{2})\\  
             &= h^{n+2}\int_{-\frac{n}{2}}^{\frac{n}{2}}\prod_{j=0}^n(u+\frac{n}{2}-j)du \quad (Moving\_towards\_left)\\
             &= 0 \quad (Odd\_function)
    \end{aligned}
    $$
    






    \subsection{梯形算法外推化}
        将区间[a,b]n等分,再将n个区间二等分,得到$\{x_k,x_{k +\frac{1}{2}}\}$ \\
        使用复合梯形公式得到:
        \[
            I = \frac{h}{4}[f(x_k) + f(x_k+\frac{1}{2}]
        \]   
        
        
\end{spacing}        
\end{CJK*}
\end{document}
